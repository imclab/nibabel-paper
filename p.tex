%-*- mode: latex; fill-column: 70 -*-
% ex: set sts=4 ts=4 sw=4 et tw=70:

\documentclass[twoside=false,fontsize=11pt,pointednumbers,normalheadings,%
               abstracton,paper=a4,pagesize,pdftex]{scrartcl}

% we need umlauts in the refs
\usepackage[utf8]{inputenc}

% serif section headings
\setkomafont{sectioning}{\normalcolor\bfseries}
\usepackage{authblk}
\renewcommand\Affilfont{\itshape\small}
\usepackage{graphicx}
\usepackage{url}
% citations
\usepackage{apacite}
\usepackage[longnamesfirst,sort&compress]{natbib}
% for code snippets
\usepackage{fancyvrb}
\usepackage{color}

\usepackage[hyperfootnotes=false,
pdftex,
pdftitle={},
pdfauthor={},
pdfsubject={},
pdfkeywords={},
pdfduplex=DuplexFlipLongEdge,
pdffitwindow=true,
pdfdisplaydoctitle=true,
debug=false,
final=true,
hyperfigures=true,
pdfborder={0 0 0}]{hyperref}

\urlstyle{same}
% howto cite projects
\newcommand{\purl}[2]{#1\footnote{\url{#2}}}

%% Miscellaneous Latin abbreviations emphasizing
%% according to
%% http://en.wikipedia.org/wiki/List_of_Latin_phrases_(C-E)#endnote_egie
%% ``American style guides tend to recommend that "e.g."  and "i.e."
%% should generally be followed by a comma, just as "for example" and
%% "that is" would be; UK style tends to omit the comma''
\newcommand{\ie}[0]{\emph{i.e.},\ }
\newcommand{\eg}[0]{\emph{e.g.},\ }
\newcommand{\etc}[0]{\emph{etc.}}

%% Unified references
\newcommand{\fig}[1]{{Figure~\ref{fig:#1}}}

% try to put figures and tables close to where they are defined
% as recommended by GUIDE
\makeatletter
\renewcommand{\fps@table}{htbp}
\renewcommand{\fps@figure}{htbp}
\makeatother

\title{NiBabel - Access a cacophony of neuro-imaging file formats}

% if you work on this manuscript, add yourself and your affiliation here
\author[1]{Matthew~Brett}
\author[2]{Michael~Hanke}
\author[3]{Stephan~Gerhard}

\affil[1]{Helen Wills Neuroscience Institute, University of California at
Berkeley, USA}
\affil[2]{Department of Experimental Psychology, University of Magdeburg,
Magdeburg, Germany}
\affil[3]{Institute of Neuroinformatics, University Zurich and Swiss Federal
Institute of Technology, Zurich, Switzerland}

% for code snippets
\input{code.sty}

\begin{document}
\maketitle

% 5 words
\texttt{Running title: Invent me}

\section*{Correspondence}
Matthew Brett\\
Wherever he may roam\\
\url{matthew.brett@gmail.com}

\section*{Acknowledgements}
Thanks the NumPy, ScipPy, \ldots folks.

\newpage

%%%%%%%%%%%%%%%%%%%%%%%%%%%%%%%%%%%%%%%%%%%%%%%%%%%%%%%%%%%%%%%%%%%%%%%%%%%%%%
%%%%%%%%%%%%%%%%%%%%%%%%%%%%%%%%%%%%%%%%%%%%%%%%%%%%%%%%%%%%%%%%%%%%%%%%%%%%%%

% figure out a target journal and the required structure of the manuscript

\section{Introduction}

\begin{itemize}
 \item historical background (PyNifti)
 \item neuroimaging IO library as basic to build the Python neuroimaging tool stack
 \item embeddedness in the NIPY project, collaborative effort
 \item exposing NumPy arrays
\end{itemize}

\section{Best Practices}

\subsection{Collaborative Software Development}

Git, GitHub, Code Reviews, Mailinglist, ...

\subsection{Test Suite}

Code coverage

\subsection{Distribution}

Release managment, (Neuro)Debian/PyPI, Documentation, ...

\section{Supported File Formats}

\subsection{NifTI1}

\subsection{ANALYZE}

\subsection{GiFTI}

\subsection{PAR/REC}

\subsection{MINC}

\subsection{TrackVis}

\subsection{Freesurfer}

\subsection{Data Format Tools}

\subsection{DICOM}


\section{Usage in software projects}

Software projects that depend on or use Nibabel.

Connectome Mapper
Connectome Viewer
Data Format Tools
DiPy
NiPy
NiPyPE
Nitime
PyHRF
PyLocator
PyMVPA \citep{HHS+09b}
pyPET
PyROI
PySurfer

\section{Future development}

\subsection{NifTI2}

\subsection{CIFTI}


\section{Code examples}
% demo code snippet
\input{img_access.py.tex}

%\bibliographystyle{abbrvnat}
\bibliographystyle{apacite}
\bibliography{references}

\end{document}
